\documentclass[a4paper]{article}

\usepackage{fancyhdr}
\pagestyle{fancy}
\lhead{Ory Band \texttt{300479425}, Uri Menkes \texttt{111111111}}
\rhead{\thepage}
\renewcommand{\headrulewidth}{0.4pt}
\renewcommand{\footrulewidth}{0.4pt}

\usepackage{amsmath,amssymb,bbm}
\newcommand{\lc}{\left\{}
\newcommand{\rc}{\right\}}
\renewcommand{\notin}{\not\in}
\newcommand{\E}{\mathbb{E}}
\newcommand{\I}{\mathbbm{1}}
\newcommand{\Sum}{\sum\limits_{i=1}^N}
\newcommand{\qed}{\tag*{$\square$}}
\newcommand{\fn}{\frac{1}{N}}

\usepackage{cancel}

\title{Assignment 1}
\author{Ory Band \texttt{300479425}\\
\and Uri Menkes \texttt{111111111}}
\date{\today}

\begin{document}

\maketitle
\newpage

\section {Question 1}
\subsection {(a)}

First, we'll notice that we're discussing discrete distributions (and not absoloutly continuous ones, for example).
Therefore, since $ \forall 1 \leq i \leq N : X_i \sim P $, we get that $ \Pr(x_i=i)=p_i $.

According to the description, $ U_m $ is the sum of $ A = \lc [N] \setminus \lc X_1, X_2, .., X_m \rc \rc $.
That is, the sum of points $ A = \lc w_1, w_2, .., w_t \rc \subseteq \lc p_1, p_2, .., p_N \rc , t \leq N $
s.t. $ \forall 1 \leq j \leq t : w_j \notin \lc X_1, X_2, .., X_m \rc $.

Therefore, using the \textit{linearity of expectation}, we can write:
\begin{align*}
    \E[U_m] = \E \Bigg[ \Sum p_i \cdot \I_{ \lc U_m=i \rc } \Bigg]
    &= \Sum p_i \cdot \E \Big[ \I_{ \lc X_1 \neq i \wedge X_2 \neq i \wedge .. \wedge X_m\neq i \rc } \Big] \\
    &= \Sum p_i \cdot \E \Big[ \Pr(X_1 \neq i \wedge X_2 \neq i \wedge .. \wedge X_m\neq i) \Big]
\end{align*}
And since $ X_1, X_2, .., X_m $ are independent, and $ (1-p_i)^m $ is constant, we get:
\begin{align*}
    &= \Sum p_i \cdot \E \Big[ \Pr(X_1 \neq i) \cdot \Pr(X_2 \neq i) \cdot .. \cdot \Pr(X_m \neq i) \Big] \\
    &= \Sum p_i \cdot \E \Big[ (1-p_i)^m \Big] \\
    &= \Sum p_i \cdot (1-p_i)^m \\
    \qed
\end{align*}

\subsection {(b)}

\subsubsection {First sub-question}

Notice that $ \forall x \in \mathbb{R} : 1-x \leq e^{-x} $ . Therefore:
\begin{align*}
    \E [ U_m ] = \Sum \fn \cdot (1-\fn)^m = \cancel{N} \cdot \cancel{\fn} \cdot (1-\fn)^m \leq (e^\fn)^m = e^{\frac{m}{N}}
\end{align*}

\subsubsection {Second sub-question}
Now,

\end{document}
